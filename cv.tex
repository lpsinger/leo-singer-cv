%%%%%%%%%%%%%%%%%%%%%%%%%%%%%%%%%%%%%%%%%
% Compact Academic CV
% LaTeX Template
% Version 1.0 (10/6/2012)
%
% This template has been downloaded from:
% http://www.LaTeXTemplates.com
%
% Original author:
% Dario Taraborelli (http://nitens.org/taraborelli/home)
%
% License:
% CC BY-NC-SA 3.0 (http://creativecommons.org/licenses/by-nc-sa/3.0/)
%
% Important:
% This template needs to be compiled using XeLaTeX
%
% Note: this template has the option to use the Hoefler Text font, see the
% font configurations section below for instructions on using this font
%
%%%%%%%%%%%%%%%%%%%%%%%%%%%%%%%%%%%%%%%%%

%----------------------------------------------------------------------------------------
%	PACKAGES AND OTHER DOCUMENT CONFIGURATIONS
%----------------------------------------------------------------------------------------

\documentclass[11pt, letterpaper]{article} % Document font size and paper size

\usepackage{fontspec} % Allows the use of OpenType fonts

\usepackage{geometry} % Allows the configuration of document margins
\geometry{a4paper, textwidth=5.5in, textheight=8.5in, marginparsep=7pt, marginparwidth=.6in} % Document margin settings
\setlength\parindent{0in} % Remove paragraph indentation

\usepackage[usenames,dvipsnames]{xcolor} % Custom colors

\usepackage{sectsty} % Allows changing the font options for sections in a document
\usepackage[normalem]{ulem} % Custom underlining
\usepackage{xunicode} % Allows generation of unicode characters from accented glyphs
\defaultfontfeatures{Mapping=tex-text} % Converts LaTeX specials (``quotes'' --- dashes etc.) to unicode

\usepackage{marginnote} % For margin years
\newcommand{\years}[1]{\marginnote{\scriptsize #1}} % New command for including margin years
\renewcommand*{\raggedleftmarginnote}{}
\setlength{\marginparsep}{7pt} % Slightly increase the distance of the margin years from the contant
\reversemarginpar

%\usepackage[xetex, bookmarks, colorlinks, breaklinks, pdftitle={Albert Einstein - vita},pdfauthor={Albert Einstein}]{hyperref} % PDF setup - set your name and the title of the document to be incorporated into the final PDF file meta-information
%\hypersetup{linkcolor=blue,citecolor=blue,filecolor=black,urlcolor=MidnightBlue} % Link colors
\usepackage{hyperref}

%----------------------------------------------------------------------------------------
%	FONT CONFIGURATIONS
%----------------------------------------------------------------------------------------

% Two font choices are available in this template, the default is Linux Libertine, available for free at: http://www.linuxlibertine.org while the secondary choice is Hoefler Text which comes bundled with Mac OSX.
% To use Hoefler Text, comment out the Linux Libertine block below and uncomment the Hoefler Text block. You will also need to replace the "\&" characters with "\amper{}" in section titles.

% Linux Libertine Font (default)
% \setromanfont [Ligatures={Common}, Numbers={OldStyle}, Variant=01]{Linux Libertine O} % Main text font
% %\setmonofont[Scale=0.8]{Monaco} % Set mono font (e.g. phone numbers)
% \sectionfont{\mdseries\upshape\Large} % Set font options for sections
% \subsectionfont{\mdseries\scshape\normalsize} % Set font options for subsections
% \subsubsectionfont{\mdseries\upshape\large} % Set font options for subsubsections
% \chardef\&="E050 % Custom ampersand character

% Hoefler Text Font (bundled with Mac OSX)
\setromanfont [Ligatures={Common}, Numbers={OldStyle}]{Hoefler Text} % Main text font
\setmonofont[Scale=0.8]{Monaco} % Set mono font (e.g. phone numbers)
\setsansfont[Scale=0.9]{Optima Regular} % Set sans font, used in the main name and titles in the document
\newcommand{\amper}{{\fontspec[Scale=.95]{Hoefler Text}\selectfont\itshape\&}} % Custom ampersand character
\sectionfont{\sffamily\mdseries\large\underline} % Set font options for sections
\subsectionfont{\rmfamily\mdseries\scshape\normalsize} % Set font options for subsections
\subsubsectionfont{\rmfamily\bfseries\upshape\normalsize} % Set font options for subsubsections

%----------------------------------------------------------------------------------------

\begin{document}

%----------------------------------------------------------------------------------------
%	CONTACT AND GENERAL INFORMATION SECTION
%----------------------------------------------------------------------------------------

{\LARGE Leo Singer}\\[1cm] % Your name
California Institute of Technology\\ % Your address
1200 E. California Blvd., MC 100-36\\
Pasadena, CA 91101
U.S.A.\\[.2cm]
Phone: 301-633-9322\\ % Your phone number
Email: \href{mailto:lsinger@caltech.edu}{lsinger@caltech.edu}\\ % Your email address
%\textsc{url}: \href{http://www.ias.edu/spfeatures/einstein/}{http://www.ias.edu/spfeatures/einstein/}\\ % Your academic/personal website

\vfill % Whitespace between contact information and specific CV information

%------------------------------------------------

Born: August 16, 1986---Rolla, Missouri, U.S.A.\\ % Your date of birth
Nationality: American\\ % Your nationality
Languages: English, French

%------------------------------------------------

\section*{Current position}

Graduate Student, California Institute of Technology % Your current or previous employment position

%------------------------------------------------

\section*{Areas of specialization}

Physics; Astronomy; Gravitational Waves; Optical Transients; Bayesian Inference; Gamma-ray Bursts. % Your primary areas of research interest

%----------------------------------------------------------------------------------------
%	EDUCATION SECTION
%----------------------------------------------------------------------------------------

\section*{Education}

\years{2009}\textsc{MSc} in Physics, University of Maryland\\
\years{2013\\(anticipated)}\textsc{PhD} in Physics, California Institute of Technology

%----------------------------------------------------------------------------------------
%	GRANTS, HONORS AND AWARDS SECTION
%----------------------------------------------------------------------------------------

\section*{Honors \& awards}

\years{2010}Graduate Research Fellow, National Science Foundation\\
\years{2012}Best Poster, LSC-Virgo Collaboration Meeting, Rome\\

%----------------------------------------------------------------------------------------
%	PUBLICATIONS AND TALKS SECTION
%----------------------------------------------------------------------------------------

\section*{Publications \& talks}

\subsection*{Journal articles}

\years{2012}Cannon, K., Cariou, R., Chapman, A., Crispin-Ortuzar, M., Fotopoulos, N., Frei, M., Hanna, C., Kara, E., Keppel, D., Liao, L., Privitera, S., Searle, A., \textbf{Singer, L.} and Weinstein, A., “Toward Early-warning Detection of Gravitational Waves from Compact Binary Coalescence", \emph{Astrophysical Journal} 748: 136\\
\years{2013}Dietz, A., Fotopoulos, N., \textbf{Singer, L.}, Cutler, C., “Outlook for detection of GW inspirals by GRB-triggered searches in the advanced detector era", \emph{Physical Review D} 87: 064033\\
\years{2013}Robitaille, T., Tollerud, E., Greenfield, P., Droettboom, M., Bray, E., Aldcroft, T., Davis, M., Ginsburg, M., Price-Whelan, A., Kerzendorf, W., Conley, A., Crighton, N., Barbary, K., Muna, D., Ferguson, H., Grollier, F., Parikh, M., Nair, P., Günther, H., Deil, C., Woillez, J., Conseil, S., Kramer, R., Turner, J., \textbf{Singer, L.}, Fox, R., Weaver, B., Zabalza, V., Edwards, Z., Bostroem, K., Burke, D., Casey, A., Crawford, S., Dencheva, N., Ely, J., Jenness, T., Labrie, K., Lim, P. L., Pierfederici, F., Pontzen, A., Ptak, A., Refsdal, B., Servillat, M. and Streicher, O., “Astropy: A community Python package for astronomy", \emph{Astronomy \& Astrophysics} 558: A33\\
\years{2013}\textbf{Singer, L.}, Cenko, S., Kasliwal, M., Perley, D. et al., “Discovery and Redshift of an Optical Afterglow in 71 deg$^2$: iPTF13bxl and GRB 130702A", \emph{Astrophysical Journal Letters} 776: 34\\
\years{2013}\textbf{Singer, L.}, Price, L., Urban, A. and Pankow, C., “The First Two Years of Gravitational-Wave Astronomy with Advanced LIGO and Virgo", \emph{in preparation}\\
\years{2013}\textbf{Singer, L.} and Price, L., “WHOOMP! (There It Is): Rapid Bayesian position reconstruction for gravitational-wave transients", \emph{in preparation}\\

\subsection*{Astronomical Circulars and Bulletins}

\years{2013}\textbf{Singer, L.}, Cenko, S. B., and Brown, D., “GRB 130122A: PTF P48 optical upper limits", \emph{GCN Circulars} 14192\\
\years{2013}\textbf{Singer, L.}, Cenko, S. B., and Brown, D., “GRB 130215A: PTF P48 optical detection", \emph{GCN Circulars} 14244\\
\years{2013}\textbf{Singer, L.}, Cenko, S. B., and Brown, D., “GRB 130305A: PTF P48 optical upper limits", \emph{GCN Circulars} 14313\\
\years{2013}\textbf{Singer, L.}, Cenko, S. B., Kasliwal, M., Brown, D., Yaron, O., Bellm, E., Caudill, S., Tinyanont, S., Khatami, D. and Weinstein, J., “Fermi394416326: iPTF detection of a possible optical afterglow", \emph{GCN Circulars} 14967\\
\years{2013}Perley, D., \textbf{Singer, L.} and Cenko, S. B., “GRB 130702A: Continued P60 observations", \emph{GCN Circulars} 14981\\
\years{2013}Kasliwal, M., \textbf{Singer, L.} and Cenko, S. B., “Fermi403206457: iPTF detection of a possible optical afterglow", \emph{GCN Circulars} 15324

%------------------------------------------------

%----------------------------------------------------------------------------------------
%	TEACHING SECTION
%----------------------------------------------------------------------------------------

\section*{Teaching}

...

\vfill{} % Whitespace before final footer

%----------------------------------------------------------------------------------------
%	FINAL FOOTER
%----------------------------------------------------------------------------------------

\begin{center}
{\scriptsize Last updated: \today\- •\- \href{http://www.LaTeXTemplates.com}{http://www.LaTeXTemplates.com}} % Any final footer text such as a URL to the latest version of your CV, last updated date, compiled in XeTeX, etc
\end{center}

%----------------------------------------------------------------------------------------

\end{document}
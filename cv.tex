%%%%%%%%%%%%%%%%%%%%%%%%%%%%%%%%%%%%%%%%%
% Compact Academic CV
% LaTeX Template
% Version 1.0 (10/6/2012)
%
% This template has been downloaded from:
% http://www.LaTeXTemplates.com
%
% Original author:
% Dario Taraborelli (http://nitens.org/taraborelli/home)
%
% License:
% CC BY-NC-SA 3.0 (http://creativecommons.org/licenses/by-nc-sa/3.0/)
%
% Important:
% This template needs to be compiled using XeLaTeX
%
% Note: this template has the option to use the Hoefler Text font, see the
% font configurations section below for instructions on using this font
%
%%%%%%%%%%%%%%%%%%%%%%%%%%%%%%%%%%%%%%%%%

%----------------------------------------------------------------------------------------
%	PACKAGES AND OTHER DOCUMENT CONFIGURATIONS
%----------------------------------------------------------------------------------------

\documentclass[10pt, letterpaper]{article} % Document font size and paper size

% List separator symbol
\newcommand{\listsep}{$\,\cdot\,$}

% Symbol fonts (for contact info dingbats)
\usepackage{fontawesome}

\usepackage{fontspec} % Allows the use of OpenType fonts

\usepackage{geometry} % Allows the configuration of document margins
\geometry{letterpaper, textwidth=5.5in, textheight=8.5in, marginparsep=7pt, marginparwidth=.6in} % Document margin settings
\setlength\parindent{0in} % Remove paragraph indentation

\usepackage[usenames,dvipsnames]{xcolor} % Custom colors

\usepackage{sectsty} % Allows changing the font options for sections in a document
\usepackage[normalem]{ulem} % Custom underlining
\usepackage{xunicode} % Allows generation of unicode characters from accented glyphs
\defaultfontfeatures{Mapping=tex-text} % Converts LaTeX specials (``quotes'' --- dashes etc.) to unicode

\usepackage{marginnote} % For margin years
\newcommand{\years}[1]{\marginnote{\scriptsize #1}} % New command for including margin years
\renewcommand*{\raggedleftmarginnote}{}
\setlength{\marginparsep}{7pt} % Slightly increase the distance of the margin years from the content
\reversemarginpar

%\usepackage[xetex, bookmarks, colorlinks, breaklinks, pdftitle={Albert Einstein - vita},pdfauthor={Albert Einstein}]{hyperref} % PDF setup - set your name and the title of the document to be incorporated into the final PDF file meta-information
%\hypersetup{linkcolor=blue,citecolor=blue,filecolor=black,urlcolor=MidnightBlue} % Link colors
\usepackage{hyperref}

%----------------------------------------------------------------------------------------
%	FONT CONFIGURATIONS
%----------------------------------------------------------------------------------------

% Two font choices are available in this template, the default is Linux Libertine, available for free at: http://www.linuxlibertine.org while the secondary choice is Hoefler Text which comes bundled with Mac OSX.
% To use Hoefler Text, comment out the Linux Libertine block below and uncomment the Hoefler Text block. You will also need to replace the "\&" characters with "\amper{}" in section titles.

% Linux Libertine Font (default)
% \setromanfont [Ligatures={Common}, Numbers={OldStyle}, Variant=01]{Linux Libertine O} % Main text font
% \setmonofont[Scale=0.8]{Monaco} % Set mono font (e.g. phone numbers)
% \sectionfont{\mdseries\upshape\Large} % Set font options for sections
% \subsectionfont{\mdseries\scshape\normalsize} % Set font options for subsections
% \subsubsectionfont{\mdseries\upshape\large} % Set font options for subsubsections
% \chardef\&="E050 % Custom ampersand character

% Hoefler Text Font (bundled with Mac OSX)
\setromanfont [Ligatures={Common}, Numbers={OldStyle}]{Hoefler Text} % Main text font
\setmonofont[Scale=0.8]{Monaco} % Set mono font (e.g. phone numbers)
\setsansfont[Scale=0.9]{Optima Regular} % Set sans font, used in the main name and titles in the document
\newcommand{\amper}{{\fontspec[Scale=.95]{Hoefler Text}\selectfont\itshape\&}} % Custom ampersand character
\sectionfont{\sffamily\mdseries\large} % Set font options for sections
\subsectionfont{\rmfamily\mdseries\scshape\normalsize} % Set font options for subsections
\subsubsectionfont{\rmfamily\bfseries\upshape\normalsize} % Set font options for subsubsections

%----------------------------------------------------------------------------------------

\begin{document}

%----------------------------------------------------------------------------------------
%	CONTACT AND GENERAL INFORMATION SECTION
%----------------------------------------------------------------------------------------

{\LARGE Leo Singer}\\[1cm] % Your name
NASA Postdoctoral Program Fellow\\
\begin{minipage}[t][2cm][s]{8cm}
Goddard Space Flight Center, Code 661 \\ % Your address
Greenbelt, MD 20771\\[.2cm]
\end{minipage}\begin{minipage}[t][2cm][s]{5cm}
{\tiny\faEnvelopeAlt} \,\,\,\,\,\, \href{mailto:leo.p.singer@nasa.gov}{\small \texttt{leo.p.singer@nasa.gov}}\\
\textsc{m} \,\,\,\,\,\, \texttt{\small +1 301 633 9322}\\ % Your phone number
%\textsc{url}: \href{http://www.ias.edu/spfeatures/einstein/}{http://www.ias.edu/spfeatures/einstein/}\\ % Your academic/personal website
\end{minipage}
%------------------------------------------------

\section*{Areas of Specialization}

Physics \listsep{}  Astronomy \listsep{} Gravitational Waves \listsep{} Optical Transients \\
Gamma-Ray Bursts \listsep{}  Bayesian Inference \listsep{} Open-Source Astronomy Software

\section*{Education}

\years{2015} Ph.D. in Physics, California Institute of Technology\\
Thesis Title: “From \emph{Fermi} GRBs to LIGO Discoveries: The Needle in the 100\,deg$^2$ Haystack"\\[0.125cm]
%
\years{2009}B.Sc. in Physics, University of Maryland

\section*{Honors \& awards}

\years{2014--present}NASA Postdoctoral Program Fellow\\[0.125cm]
%
\years{2014}\href{https://gwic.ligo.org/thesisprize/2014/}{GWIC Thesis Prize}\\[0.125cm]
%
\years{2013--14}John and Ursula Kanel Charitable Foundation Scholar\\[0.125cm]
%
\years{2010--13}Graduate Research Fellow, National Science Foundation\\[0.125cm]
%
\years{2012}Best Poster, LSC-Virgo Collaboration Meeting, Rome

\section*{Publications \& talks}

\subsection*{Selected refereed journal articles}

\years{2015}\href{http://dx.doi.org/10.1088/0004-637X/806/1/52}{\textbf{Singer, L. P.}, Kasliwal, M. M., Cenko, S. B., Perley, D., et al., “The Needle in the 100\,deg$^2$ Haystack: Uncovering Afterglows of \emph{Fermi} GRBs with the Palomar Transient Factory", \emph{Astrophysical Journal} 806: 52}\\[0.125cm]
%
\years{2015}\href{http://dx.doi.org/10.1088/0004-637X/804/2/114}{Berry, C. P. L., Mandel, I., Middleton, H., \textbf{Singer, L. P.}, et al., “Parameter Estimation for Binary Neutron-Star Coalescences with Realistic Noise During the Advanced LIGO Era", \emph{Astrophysical Journal} 804: 114}\\[0.125cm]
%
\years{2014}\href{http://dx.doi.org/10.1088/0004-637X/795/2/105}{\textbf{Singer, L. P.}, Price, L. R., Farr, B., et al., “The First Two Years of Electromagnetic Follow-Up with Advanced LIGO and Virgo", \emph{Astrophysical Journal} 795: 105}\\[0.125cm]
%
\years{2014}\href{http://dx.doi.org/10.1103/PhysRevD.89.084060}{Sidery, T., Aylott, B., Christensen, N., Farr, B., Farr, W., Feroz, F., Gair, J., Grover, K., Graff, P., Hanna, C., Kalogera, V., Mandel, I., O'Shaughnessy, R., Pitkin, M., Price, L., Raymond, V., \textbf{Singer, L. P.}, et al., “Reconstructing the Sky Location of Gravitational-Wave Detected Compact Binary Systems: Methodology for Testing and Comparison", \emph{Physical Review D} 89: 084060}\\[0.125cm]
%
\years{2013}\href{http://dx.doi.org/10.1088/2041-8205/776/2/L34}{\textbf{Singer, L. P.}, Cenko, S., Kasliwal, M., Perley, D. et al., “Discovery and Redshift of an Optical Afterglow in 71 deg$^2$: iPTF13bxl and GRB 130702A", \emph{Astrophysical Journal Letters} 776: 34}\\[0.125cm]
%
\years{2013}\href{http://dx.doi.org/10.1051/0004-6361/201322068}{Robitaille, T., Tollerud, E., Greenfield, P., Droettboom, M., Bray, E., Aldcroft, T., Davis, M., Ginsburg, M., Price-Whelan, A., Kerzendorf, W., Conley, A., Crighton, N., Barbary, K., Muna, D., Ferguson, H., Grollier, F., Parikh, M., Nair, P., Günther, H., Deil, C., Woillez, J., Conseil, S., Kramer, R., Turner, J., \textbf{Singer, L. P.}, et al., “Astropy: A Community Python Package for Astronomy", \emph{Astronomy \& Astrophysics} 558: A33}\\[0.125cm]
%
\years{2013}\href{http://dx.doi.org/10.1103/PhysRevD.87.064033}{Dietz, A., Fotopoulos, N., \textbf{Singer, L. P.}, Cutler, C., “Outlook for detection of GW inspirals by GRB-Triggered Searches in the Advanced Detector Era", \emph{Physical Review D} 87: 064033}\\[0.125cm]
%
\years{2012}\href{http://dx.doi.org/10.1088/0004-637X/748/2/136}{\textbf{(note: corresponding author)} Cannon, K., Cariou, R., Chapman, A., Crispin-Ortuzar, M., Fotopoulos, N., Frei, M., Hanna, C., Kara, E., Keppel, D., Liao, L., Privitera, S., Searle, A., \textbf{Singer, L. P.} and Weinstein, A., “Toward Early-warning Detection of Gravitational Waves from Compact Binary Coalescence", \emph{Astrophysical Journal} 748: 136}

% \subsection*{Large collaboration journal articles}
% 
% \years{2013}\href{http://dx.doi.org/10.1103/PhysRevD.88.102002}{“A directed search for continuous Gravitational Waves from the Galactic Center", \emph{Physical Review} D88: 102002}\\[0.125cm]
% %
% \years{2013}\href{http://dx.doi.org/10.1038/nphoton.2013.177}{“Enhancing the sensitivity of the LIGO gravitational wave detector by using squeezed states of light", \emph{Nature Photonics} 7: 613-619}\\[0.125cm]
% %
% \years{2013}\href{http://dx.doi.org/10.1103/PhysRevD.88.062001}{“Parameter estimation for compact binary coalescence signals with the first generation gravitational-wave detector network", \emph{Physical Review} D88: 062001}\\[0.125cm]
% %
% \years{2013}\href{http://dx.doi.org/10.1103/PhysRevD.87.022002}{“Search for Gravitational Waves from Binary Black Hole Inspiral, Merger and Ringdown in LIGO-Virgo Data from 2009-2010", \emph{Physical Review} D87: 022002}\\[0.125cm]
% %
% \years{2013}\href{http://dx.doi.org/10.1103/PhysRevD.87.042001}{“Einstein@Home all-sky search for periodic gravitational waves in LIGO S5 data", \emph{Physical Review} D87: 042001}\\[0.125cm]
% %
% \years{2012}\href{http://dx.doi.org/10.1088/0004-637X/760/1/12}{“Search for gravitational waves associated with gamma-ray bursts during LIGO science run 6 and Virgo science runs 2 and 3", \emph{Astrophysical Journal} 760: 12}\\[0.125cm]
% %
% \years{2012}\href{http://dx.doi.org/10.1088/0067-0049/203/2/28}{“Swift follow-up observations of candidate gravitational-wave transient events", \emph{Astrophysical Journal Supplements} 203: 28}\\[0.125cm]
% %
% \years{2012}\href{http://dx.doi.org/10.1088/0264-9381/29/15/155002}{“The characterization of Virgo data and its impact on gravitational-wave searches", \emph{Classical \& Quantum Gravity} 29: 155002}\\[0.125cm]
% %
% \years{2012}\href{http://dx.doi.org/10.1103/PhysRevD.85.122007}{“All-sky search for gravitational-wave bursts in the second joint LIGO-Virgo run", \emph{Physical Review} D85: 122007}\\[0.125cm]
% %
% \years{2012}\href{http://dx.doi.org/10.1103/PhysRevD.85.102004}{“Search for Gravitational Waves from Intermediate Mass Binary Black Holes", \emph{Physical Review} D85: 102004}\\[0.125cm]
% %
% \years{2012}\href{http://dx.doi.org/10.1088/0004-637X/755/1/2}{“Implications For The Origin Of GRB 051103 From LIGO Observations", \emph{Astrophysical Journal} 755: 2}\\[0.125cm]
% %
% \years{2012}\href{http://dx.doi.org/10.1103/PhysRevD.85.122001}{“Upper limits on a stochastic gravitational-wave background using LIGO and Virgo interferometers at 600-1000 Hz", \emph{Physical Review} D85: 122001}\\[0.125cm]
% %
% \years{2012}\href{http://dx.doi.org/10.1103/PhysRevD.85.082002}{“Search for Gravitational Waves from Low Mass Compact Binary Coalescence in LIGO's Sixth Science Run and Virgo's Science Runs 2 and 3", \emph{Physical Review} D85: 082002}\\[0.125cm]
% %
% \years{2012}\href{http://dx.doi.org/10.1103/PhysRevD.85.022001}{“All-sky Search for Periodic Gravitational Waves in the Full S5 LIGO Data", \emph{Physical Review} D85: 022001}\\[0.125cm]
% %
% \years{2012}\href{http://dx.doi.org/10.1051/0004-6361/201118219}{“Implementation and testing of the first prompt search for gravitational wave transients with electromagnetic counterparts", \emph{Astronomy \& Astrophysics} 539: A124}\\[0.125cm]
% %
% \years{2011}\href{http://dx.doi.org/10.1088/0004-637X/737/2/93}{“Beating the spin-down limit on gravitational wave emission from the Vela pulsar", \emph{Astrophysical Journal} 737: 93}\\[0.125cm]
% %
% \years{2011}\href{http://dx.doi.org/10.1088/2041-8205/734/2/L35}{“Search for Gravitational Wave Bursts from Six Magnetars", \emph{Astrophysical Journal Letters} 734: L35}

\subsection*{Journal articles in preparation}

\years{2015}\textbf{Singer, L.} et al., “Going the Distance: Rapid Three-Dimensional Position Estimation for Correlating Advanced LIGO Sources with Galaxies in the Local Universe"\\[0.125cm]
%
\years{2015}\href{http://arxiv.org/abs/1508.03608}{Gehrels, N., Cannizzo, J. K., Kanner, J., Kasliwal, M. M., Nissanke, S., \textbf{Singer, L.}, “Galaxy Strategy for LIGO-Virgo Gravitational Wave Counterpart Searches", submitted to \emph{Astrophysical Journal}}\\[0.125cm]
%
\years{2015}\href{http://arxiv.org/abs/1508.03634}{\textbf{Singer, L.} and Price, L., “WHOOMP! (There It Is): Rapid Bayesian Position Reconstruction for Gravitational-Wave Transients", submitted to \emph{Physical Review D}}

\subsection*{Invited talks}
%
\years{2015}“Compact Binary Mergers in the Era of Advanced LIGO", \emph{General Relativity \& Gravitation: A Centennial Perspective}, Pennsylvania State University, State College, PA\\[0.125cm]
%
\years{2015}“Gravitational Wave Observations and Optical Follow-up with Advanced LIGO", \emph{Improving Data Mobility \& Management for International Cosmology}, Lawrence Berkeley National Lab, Berkeley, CA\\[0.125cm]
%
\years{2013}“The Needle in the Hundred-Square-Degree Haystack: from Fermi GRBs to LIGO Discoveries", \emph{Hot-Wiring the Transient Universe III}, Santa Fe, NM\\[0.125cm]
%
\years{2013}“Relativistic Explosions with Palomar Transient Factory", \emph{Gamma-ray Bursts: New Missions to New Science}, Skobeltsyn Institute of Nuclear Physics of Moscow State University, Moscow, Russia \\[0.125cm]
%
\years{2013}“HTCondor in MacPorts", \emph{HTCondor Week}, University of Wisconsin, Madison, WI

\subsection*{Selected campus \& departmental talks}
%
\years{2015}“From Fermi GRBs to LIGO Discoveries: The Needle in the 100-Square-Degree Haystack", Carnegie Observatories, Pasadena, CA\\[0.125cm]
%
\years{2015}“From Fermi GRBs to LIGO Discoveries: The Needle in the 100-Square-Degree Haystack", \emph{LCOGT Seminar Series}, Las Cumbres Observatory Global Telescope Network, Goleta, CA\\[0.125cm]
%
\years{2015}“From \emph{Fermi} GRBs to LIGO Discoveries: The Needle in the 100~deg$^2$ Haystack", \emph{National Space Science \& Technology Center Space Science Seminar}, University of Alabama, Hunstville, AL\\[0.125cm]
%
\years{2013}“The Rubber Meets the Road: Discovery and Redshift of an Optical Afterglow in 71 deg$^2$", \emph{Intermediate Palomar Transient Factory Workshop}, California Institute of Technology, Pasadena, CA\\[0.125cm]
%
\years{2013 \& 2014}“Astropy Boot Camp", \emph{Intermediate Palomar Transient Factory Workshop}, California Institute of Technology, Pasadena, CA\\[0.125cm]
%
\years{2013}“The Road to Multimessenger Astronomy with Advanced LIGO", \emph{LIGO Seminar}, California Institute of Technology, Pasadena, CA\\[0.125cm]
%
\years{2011}“Toward Early-warning Detection of Gravitational Waves from Compact Binary Coalescence", \emph{LIGO Seminar}, California Institute of Technology, Pasadena, CA

\subsection*{Astronomical circulars \& bulletins}

\years{2014}\href{http://gcn.gsfc.nasa.gov/gcn3/16497.gcn3}{Perley, D. A. and \textbf{Singer, L.}, Kasliwal, M., Bhalerao, V., Cenko, S. B., Cao, Y., Duggan, G., Perley, D., and Johansson, J., “Fermi 425667201: optical counterpart search", \emph{GCN Circulars} 16497}\\[0.125cm]
%
\years{2014}\href{http://gcn.gsfc.nasa.gov/gcn3/16362.gcn3}{Perley, D. A. and \textbf{Singer, L.}, “Fermi 423717114: Additional P48 and P60 observations", \emph{GCN Circulars} 16362}\\[0.125cm]
%
\years{2014}\href{http://gcn.gsfc.nasa.gov/gcn3/16360.gcn3}{\textbf{Singer, L.}, Kasliwal, M., and Cenko, S. B., “Fermi423717114: iPTF optical transient candidates", \emph{GCN Circulars} 16360}\\[0.125cm]
%
\years{2014}\href{http://gcn.gsfc.nasa.gov/gcn3/16226.gcn3}{\textbf{Singer, L.}, Cenko, S. B., Kasliwal, M., Fremling, C., and Dzigan, Y., “GRB 140508A: iPTF optical transient candidate", \emph{GCN Circulars} 16226}\\[0.125cm]
%
\years{2014}\href{http://gcn.gsfc.nasa.gov/gcn3/15878.gcn3}{\textbf{Singer, L.}, Kasliwal, M. and Cenko, S. B., “GRB 140219A: iPTF optical observations", \emph{GCN Circulars} 15878}\\[0.125cm]
%
\years{2014}\href{http://www.astronomerstelegram.org/?read=5807}{Arcavi, I., Gal-Yam, A., Yaron, O., \textbf{Singer, L.}, Cao, Y., Drake, A. and Djorgovski, S., “iPTF Discovery of iPTF14ih, a Type II Supernova on the Outskirts of a Spiral Host", \emph{Astronomer's Telegram} 5807}\\[0.125cm]
%
\years{2014}\href{http://gcn.gsfc.nasa.gov/gcn3/15758.gcn3}{\textbf{Singer, L.}, Kasliwal, M., Cenko, S. B. and Cao, Y., “Fermi412093190: iPTF optical counterpart search", \emph{GCN Circulars} 15758}\\[0.125cm]
%
\years{2014}\href{http://gcn.gsfc.nasa.gov/gcn3/15696.gcn3}{\textbf{Singer, L.}, Kasliwal, M., Cenko, S. B., Bellm, E. and Cao, Y., “GRB 140105A: continued iPTF observations and rejection of optical candidates", \emph{GCN Circulars} 15696}\\[0.125cm]
%
\years{2014}\href{http://gcn.gsfc.nasa.gov/gcn3/15686.gcn3}{\textbf{Singer, L.}, Kasliwal, M. and Cenko, S. B., “Fermi410578384: iPTF optical afterglow candidates for a possible short GRB", \emph{GCN Circulars} 15686}\\[0.125cm]
%
\years{2013}\href{http://gcn.gsfc.nasa.gov/gcn3/15643.gcn3}{\textbf{Singer, L.}, Cenko, S. B. and Kasliwal, M., “GRB 131231A: iPTF optical afterglow candidate coincident with Nanshan detection", \emph{GCN Circulars} 15643}\\[0.125cm]
%
\years{2013}\href{http://gcn.gsfc.nasa.gov/gcn3/15574.gcn3}{\textbf{Singer, L.}, Kasliwal, M., Cenko, S. B., Cao, Y., Perley, D. and de Cia, A., “GRB 131127B: rejection of optical counterpart candidates", \emph{GCN Circulars} 15574}\\[0.125cm]
%
\years{2013}\href{http://gcn.gsfc.nasa.gov/gcn3/15572.gcn3}{\textbf{Singer, L.}, Kasliwal, M. and
Cenko, S. B., “GRB 131126A: iPTF upper limits on the afterglow of a short GRB", \emph{GCN Circulars} 15572}\\[0.125cm]
%
\years{2013}\href{http://gcn.gsfc.nasa.gov/gcn3/15552.gcn3}{\textbf{Singer, L.}, Kasliwal, M. and
Cenko, S. B., “GRB 131125A: iPTF upper limits on the afterglow of a short GRB", \emph{GCN Circulars} 15552}\\[0.125cm]
%
\years{2013}\href{http://gcn.gsfc.nasa.gov/gcn3/15524.gcn3}{\textbf{Singer, L.} and Kasliwal, M., “Fermi407254341: iPTF optical afterglow candidates", \emph{GCN Circulars} 15524}\\[0.125cm]
%
\years{2013}\href{http://gcn.gsfc.nasa.gov/gcn3/15324.gcn3}{Kasliwal, M., \textbf{Singer, L.} and Cenko, S. B., “Fermi403206457: iPTF detection of a possible optical afterglow", \emph{GCN Circulars} 15324}\\[0.125cm]
%
\years{2013}\href{http://gcn.gsfc.nasa.gov/gcn3/14981.gcn3}{Perley, D., \textbf{Singer, L.} and Cenko, S. B., “GRB 130702A: Continued P60 observations", \emph{GCN Circulars} 14981}\\[0.125cm]
%
\years{2013}\href{http://gcn.gsfc.nasa.gov/gcn3/14967.gcn3}{\textbf{Singer, L.}, Cenko, S. B., Kasliwal, M., Brown, D., Yaron, O., Bellm, E., Caudill, S., Tinyanont, S., Khatami, D. and Weinstein, J., “Fermi394416326: iPTF detection of a possible optical afterglow", \emph{GCN Circulars} 14967}\\[0.125cm]
%
\years{2013}\href{http://gcn.gsfc.nasa.gov/gcn3/14313.gcn3}{\textbf{Singer, L.}, Cenko, S. B., and Brown, D., “GRB 130305A: PTF P48 optical upper limits", \emph{GCN Circulars} 14313}\\[0.125cm]
%
\years{2013}\href{http://gcn.gsfc.nasa.gov/gcn3/14244.gcn3}{\textbf{Singer, L.}, Cenko, S. B., and Brown, D., “GRB 130215A: PTF P48 optical detection", \emph{GCN Circulars} 14244}\\[0.125cm]
%
\years{2013}\href{http://gcn.gsfc.nasa.gov/gcn3/14192.gcn3}{\textbf{Singer, L.}, Cenko, S. B., and Brown, D., “GRB 130122A: PTF P48 optical upper limits", \emph{GCN Circulars} 14192}

\section*{Selected service to profession}
%
\years{2015--present}Co\nobreakdashes-Chair, Electromagnetic Follow-Up Working Group, LIGO Scientific Collaboration\\[0.125cm]
%
\years{2015}Session Chair at April 2015 American Physical Society Meeting

\section*{Teaching}

\years{2010--13}Mentor for LIGO Summer Undergraduate Research Fellowship\\[0.125cm]
%
\years{2011}Teaching assistant for the course “Waves, Quantum Mechanics, and Statistical Physics", California Institute of Technology, Pasadena, CA\\[0.125cm]
%
\years{2011}Lecture for LIGO Summer Undergraduate Research Fellowship program, “Introduction to Digital Signal Processing", California Institute of Technology, Pasadena, CA\\

\section*{Related employment \& activities}

\years{2008--09}\textsc{Experimental Nuclear Physics, University of Maryland}\\
Computer modeling of light collection system for neutron electric dipole moment experiment\\[0.125cm]
\years{2007-09}\textsc{NSF Research Externship for Undergraduates, TRX Systems, Inc.}\\
Delivered orientation estimation algorithm for first responder tracking system\\
Designed temperature calibration apparatus for inertial navigation units\\
Wrote embedded navigation algorithms, filesystems, and bootloaders for microcontrollers\\[0.125cm]
\years{2007}\textsc{Institute for Physical Science and Technology, University of Maryland}\\
Designed acousto-optic dithering system for high fidelity dynamic holograms\\
Simulated laser/matter interactions in MATLAB\\
Installed high accuracy translation stage for matter optics experiments\\[0.125cm]
\years{2005-06}\textsc{Intern, Proteus Technologies, LLC}\\
Design fault-tolerant, multi-user, distributed visualizations\\
Top Secret clearance with SCI access (full scope polygraph, April 11, 2006)\\[0.125cm]
\years{2006}\textsc{Atomic, Molecular, and Optical Group, University of Maryland}\\
Supported setup of atom traps\\
Constructed high voltage supply for Fabry-Perot interferometer\\[0.125cm]
\years{2006--09}\textsc{Robotics@Maryland, University of Maryland}\\
Founder (2006), project leader (2006--08), and president (2007-08) of robotics club\\
First place, AUVSI \& ONR’s XI International Autonomous Underwater Vehicle Competition\\
Raised \$35k/year in funding from academic departments and industrial partners\\
Hosted two National Science Foundation Research Externships for Undergraduates\\

% \section*{Other qualifications}

\section*{Contributions to open-source software}

\href{http://www.astropy.org}{Astropy} \listsep{} \href{http://matplotlib.org}{Matplotlib} \listsep{} \href{http://www.scipy.org}{SciPy} \listsep{} \href{http://healpix.jpl.nasa.gov}{HEALPix} \listsep{} \href{http://www.macports.org/}{MacPorts} \listsep{} \href{https://www.debian.org}{Debian} \listsep{} \href{http://research.cs.wisc.edu/htcondor/}{HTCondor} \listsep{} \href{http://gstreamer.net}{GStreamer}

% \section*{Languages}
%
% English (native tongue) \listsep{} French (proficient)

\vfill{} % Whitespace before final footer

%----------------------------------------------------------------------------------------
%	FINAL FOOTER
%----------------------------------------------------------------------------------------

\begin{center}
{\scriptsize Last updated: \today\- •\- Based on Dario Taraborelli's LaTeX template, \href{http://www.LaTeXTemplates.com}{http://www.LaTeXTemplates.com}} % Any final footer text such as a URL to the latest version of your CV, last updated date, compiled in XeTeX, etc
\end{center}

%----------------------------------------------------------------------------------------

\end{document}
